\chapter{母集団・標本}

統計学には以下の大目標があります。
\begin{itemize}
  \item (帰納的)推論
  \item 「ノイズ」と「シグナル」の分離
  \item 意思決定とリスク管理(結局どう統計データを扱うのか?
\end{itemize}

前章では推論を行うための道具を、一つの確率変数に対していくつか紹介しました。
しかし、例えば実験誤差をどう「分離」し、結果の妥当性を判断するかという統計学の役割を果たすためには、一つの確率変数を見ているだけでは不可能です。
実験を一回だけ行って「うまくいきました!」と報告しても、その結果が様々なノイズによって偶然そうなった可能性があるため、当然リジェクトされるでしょう。
残念なことに歴史上そのようなフィルターが機能しなかったことも多くあり、後世に甚大な影響を及ぼした例は枚挙にいとまがありません。
高温超電導物質捏造事件や旧石器捏造事件、またSTAP細胞事件など。
この背景には、そういった統計学的リテラシーの低さがあると思われます(個人の感想です)。

もとい。
そういった不誠実な事件を減らすためにも、統計学はかなり中立的な方法を提示してくれています。
その土台となる数学的な定義集をこの章でも行っていきましょう。



\section{母集団}

前章では平均、分散・標準偏差、歪度、尖度を紹介しましたが、統計学においてそもそも何のためにそれらを計算するのか、という話をしていませんでした。
いや、前章でさらっと話はしていますが、測度論の言葉を使ってしっかり明確に定義していきましょう。

\begin{definition}
  確率空間$(\Omega,\mathcal{F},P)$を\textbf{母集団}と呼ぶ。
  また$\Omega$上の確率変数$X$を一つ固定するとき、その確率分布(像測度)
  \[
    \mu_X:=P\circ X^{-1}:\mathcal{B}(\mathbb{R})\to[0,1]
  \]
  を\textbf{母分布}と呼ぶ。
\end{definition}



\section{独立・従属}

サイコロを2回投げることを考えると、1回目の試行の結果は2回目の試行に影響を及ぼさないと考えることができます。
実験においても、1回目の測定によって2回目の測定に影響が及ばないよう細心の注意を払うべきでしょう\footnote{
  ここは性善説をとらざるをえないでしょう。
}。
例えば化学実験をするたびにフラスコを純水で洗う等です。
統計学では、このような\textbf{「互いに無関係である」という状況を数学的に定式化するために}、以下のような定義をします。

\begin{definition}
  $(\Omega,\mathcal{F},P)$を母集団、$X_1,X_2$を$\Omega$上の確率変数とする。
  任意のボレル集合$B_1,B_2\in\mathcal{B}(\mathbb{R})$に対して、
  \[
    P(X_1^{-1}(B_1)\cap X_2^{-1}(B_2))=P(X_1^{-1}(B_1))P(X_2^{-1}(B_2))
  \]
  を満たす時、$X_1$と$X_2$は互いに\textbf{独立}であるという。
  $X_1$と$X_2$が独立でない時、互いに\textbf{従属}しているという。
\end{definition}
$P(X_1^{-1}(B_1)\cap X_2^{-1}(B_2))=P(X_1^{-1}(B_1))P(X_2^{-1}(B_2))$は、慣例的な表記では
\[
  P(X_1 \in B_1, X_2 \in B_2) = P(X_1 \in B_1)P(X_2 \in B_2)
\]
と記述します。

例えば「国民」という母集団に対して、「身長」という確率変数$X$と、「体重」という確率変数$Y$を考えましょう。
直感的には、身長が高ければ高いほど体重も大きくなる(身長に関する情報が、体重の予測に影響を与える)という感覚があります。
これは従属の例です。



\section{共分散・相関係数}

TODO

\subsection{共分散行列}

TODO



\section{標本}

母集団から何個かデータを取ってきたものを\textbf{標本}といいます。
数学的には、母集団分布(母分布)と同じ分布に従う、独立な確率変数の列として定義されます。

\begin{definition}
  母集団$(\Omega,\mathcal{F},P)$上の確率変数$X$の分布(母分布)を$\mu$とする。
  このとき、すべて母分布$\mu$に従い、かつ互いに独立な$n$個の確率変数の組
  \[
  X_1, X_2, \dots, X_n
  \]
  を、大きさ$n$の\textbf{無作為標本}(あるいは単に\textbf{標本})と呼ぶ。
\end{definition}

実験の例では、$n$回繰り返して得られるデータ全体の組 $(X_1, \dots, X_n)$ が標本です。
個々の $X_i$ は $i$ 回目の測定における値を意味します。

国民という母集団の例では、無作為に選ばれた$n$人のアンケート結果の集まりが標本となります。
解答用紙に書かれた「身長」などの一つの数値が、それぞれの $X_i$ に対応します。

\subsection{iid}

TODO



\section{統計量}

大きさ$n$の標本$X_1,\dots,X_n$から演算して得られる一つの確率変数を\textbf{統計量}といいます。
統計学の目標とするべき代表的な統計量は以下のようなものがあります。

\begin{definition}
  $X_1,\dots,X_n$を母集団$(\Omega,\mathcal{F},P)$の標本とする。
  このとき、確率変数
  \[
    \overline{X}_n:=\frac1n\sum_{i=1}^nX_n
  \]
  を、$X_1,\dots,X_n$の\textbf{標本平均}という。

  \[
    \overline{S}_n^2:=\frac1n\sum_{i=1}^n(X_i-\overline{X}_n)
  \]
  を、$X_1,\dots,X_n$の\textbf{標本分散}という。
\end{definition}

\subsection{国民の身長の例}

$\Omega$は国民の集合で、$\omega\in\Omega$は国民ひとりひとりです。
このとき確率測度として取るべき$P$は、数え上げ測度の正規化
\[
  P(A):=\frac{\#A}{\#\Omega}
\]
が自然でしょう。
また$X$は国民$\omega$の身長を意味し、母分布の値
\[
  \mu_X([150,160])=P(150\le X\le 160)
\]
は、身長150cmから160cmであるような国民の割合を意味します。

\subsection{実験の測定誤差}

実験では、期待する「真の値」というものがあります。
これを$\mu\in\mathbb{R}$とします。

$\Omega$は、ありとあらゆる可能性を含んだ、測定の瞬間の状態の集合です。
$\omega\in\Omega$は、「ある一回の具体的な測定施行」そのものになります。
$\omega$には、その測定を行った時のノイズ(気温・気圧、手振れ等)を含んでいると考えます。

確率変数$X$は、観測$\omega$に対して観測機器に表示される値を意味します。
$X$の母分布$\mu_X$は、おそらく期待する「真の値」$\mu$をピークとする確率密度関数を持つでしょう。

実際、もし個々の実験が独立かつ同じ条件(独立同分布)であれば、\textbf{大数の法則}によって実験を繰り返せば繰り返すほどその標本平均は「真の値 $\mu$」に近づきます。
また、実験回数$n$が十分大きければ、元の母分布$\mu_X$がどのような形であっても、\textbf{中心極限定理}によって「標本平均 $\overline{X}_n$ のばらつき(真の値からの誤差)」は正規分布で近似できることが保証されます。

実験レポートでは 実験回数$n$、平均 $\overline{X}_n$、不偏分散 $s^2$を報告し、推定精度の指標(標準誤差など)を示すべきでしょう。

