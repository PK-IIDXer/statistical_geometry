\documentclass[dvipdfmx, 11pt, oneside, openany]{jsbook}

\usepackage{amsmath}
\usepackage{amsfonts}
\usepackage{amsthm}
\usepackage{amssymb}
\usepackage{comment}
\usepackage{tikz-cd}

% BibLaTeXの設定
\usepackage[style=numeric]{biblatex}
\addbibresource{references.bib} % 参考文献ファイルの指定
% 出力イメージ: Hartshorne, "Algebraic Geometry" [1]
\newcommand{\mycite}[1]{%
  \citeauthor{#1}, ``\citetitle{#1}''~\cite{#1}%
}

\theoremstyle{definition}
\newtheorem{theorem}{定理}[section]
\newtheorem{corollary}[theorem]{系}
\newtheorem{definition}[theorem]{定義}
\newtheorem{lemma}[theorem]{補題}
\newtheorem{problem}[theorem]{問題}
\newtheorem{example}[theorem]{例}
\newtheorem{conjecture}[theorem]{予想}
\newtheorem{remark}{補足}
\renewcommand{\proofname}{\textbf{証明}}

\setlength{\textwidth}{\fullwidth}

\title{情報幾何ノート}
\author{ずんだもん博士}
\date{\today}

\begin{document}
  \maketitle
  \tableofcontents
  \chapter*{前提知識一覧}

\begin{description}
  \item[ガウス曲面論] 多様体以前に、その動機となったガウスの曲面論を知っておくのは教養として良いことです。
  \begin{itemize}
    \item Kobayashi, \textit{Differential geometry of curves and surfaces} \cite{Ko19}\\
    (なぜか日本語版がCiNiiで見つからなかったよ。日本語版もあるから安心してね。)
    \item Singer \& Thorpe (翻訳: 松江, 一楽), \textbf{トポロジーと幾何学入門}\cite{Sin-Tho76} (第7,8章)
  \end{itemize}
  \item[多様体論] 以下の本は鉄板ですね。
  なにも言いますまい。四の五の言わずに全部読んでください。
  \begin{itemize}
    \item 松島, \textbf{多様体入門} \cite{Matsushima17}
    \item 松村, \textbf{多様体の基礎} \cite{Matsumoto88}
  \end{itemize}
  \item[代数トポロジー] 多様体論を経た学生には是非Bott\&Tuをおすすめします。
  Hatcherは英語ですが分かりやすく、無料でPDFが配布されているため非常におすすめです。
  このあたりから普通に英語が読めるようになっておいた方がいいです。
  \begin{itemize}
    \item Bott \& Tu, \textit{Differential forms in algebraic topology} \cite{BT82}
    \item Hatcher, \textit{Algebraic topology} \cite{Hat01}
  \end{itemize}
  \item[Riemann幾何] 上記の書籍にもある程度Riemann幾何が解説されていますが、辞書的な本をさらっと読んでおくといいでしょう。
  Riemann幾何は相対論と両輪で理解するとはかどりますので、一冊紹介しましょう。
  \begin{itemize}
    \item 酒井, \textit{リーマン幾何学} \cite{Sak92}
    \item Schutz, (翻訳: 江里口, 二間瀬), \textit{シュッツ相対論入門} \cite{Sch23}
  \end{itemize}
\end{description}

  \part{情報幾何学基礎}
  \chapter{確率分布がなす多様体}

皆さんは当然、\textbf{正規分布}、みたことあると思います。
\[
  p(x;\mu,\sigma)=\frac{1}{\sqrt{2\pi}\sigma}\exp\left\{-\frac{(x-\mu)^2}{2\sigma^2}\right\}
\]
これは平均$\mu\in\mathbb{R}$と分散$\sigma>0$が定まれば、一つの正規分布が定まると見ることができます。
このように、\textbf{確率分布をパラメータづける空間}を幾何的に調べてみましょう、というのが素朴な出発点のようです。
確率分布$p(x)$が$\xi\in\mathbb{R}^n$でパラメータづけることができる場合、これを
\[
  p(x;\xi)
\]
と書きます。

もちろんこれだけでは、好き勝手に$\mathbb{R}^n$の開集合を考えてるだけにすぎないです。
ここに\textbf{Fisher情報量}あるいは\textbf{Fisher情報行列}と呼ばれる構造を含めて考えると、これがそのままRiemann計量になるという指摘が古くからなされています\footnote{
  TODO: 出典?Information and the Accuracy Attainable in the Estimation of Statistical Parameters
}。
ものとしては、エントロピー$\log p(x;\xi)$を$\xi^i$で微分したものの積の、期待値です。
\[
  g_{ij}(\xi):=E_\xi\left[\frac{\partial \log p(x;\xi)}{\partial \xi^i}\frac{\partial \log p(x;\xi)}{\partial \xi^j}\right]
\]

Riemann幾何を学んだ人は、計量があればそれに付随するRiemann接続を考えたくなります。
具体的には、統計的モデルの実際の分布関数を推定するために、いくつかのサンプルを観測して特徴量をあてはめるわけですが、その偏り具合・バイアスがほぼそのまま接続としての役割を果たすようです。
ところが統計学から自然に出てくる接続$\nabla^{(\alpha)}$ ($\alpha$は任意の実数)は、一般にはRiemann接続にはなっていません。
ところがベクトル場$X,Y,Z$に対して
\[
  Xg(Y,Z)=g(\nabla_X^{(\alpha)}Y,Z)+g(Y,\nabla_X^{(-\alpha)}Z)
\]
を満たすという意味で、$\nabla^{(\alpha)}$と$\nabla^{(-\alpha)}$は\textbf{双対接続}となっています。
また$\nabla^{(\alpha)}$の接続係数は、Riemann接続$\nabla=nabla^{(0)}$の接続係数$\Gamma_{ij,k}^{(0)}$によって
\[
  \Gamma_{ij,k}^{(\alpha)}(\xi)=\Gamma_{ij,k}^{(0)}+\frac{1-\alpha}{2}E_{\xi}[\partial_i\log p\partial_j\log p\partial_k\log p]
\]
とあらわされるものですから、これらをもとに微分幾何をやろうというのは非常に自然な発想と言えましょう。





\section{Fisher情報量}

Fisher情報量とは一言でいうと「推定のしやすさをあらわす量」です。
つまり、Fisher情報量が大きければ大きいほど、少ないサンプルから妥当な推定ができると言え、Fisher情報量が小さければ小さいほど、妥当な推定を行うために多くのサンプルが必要となると言える量です。
例えば正規分布であれば、分散が小さければ小さいほど山が尖っており、少ないサンプルでもおおよそ「どこにピークがあるか?」を探しやすいです。
これはラジオのチューニングにたとえることができるでしょう\footnote{
  今はそもそもラジオを聞く人がいないかもしれませんが、昔はアナログなツマミを回して目当ての電波を拾っていました。
}。

数学的には、以下のように厳密に定義していきます。

\begin{definition}[統計的モデル]
  $(\Omega,\mathcal{B})$を測度空間とする。
  また、$n>0$を整数とし、$\Xi$を$\mathbb{R}^n$の部分集合とする。
  $\Omega$上の確率密度の族$S$が$\Xi$でパラメータ付けられる場合、即ち
  \[
    S=\{p_{\xi}=p(x;\xi)\mid \xi=(\xi^1,\dots,\xi^n)\in\Xi\}
  \]
  と書けており、かつ対応
  \[
    \Xi\to S;\xi\mapsto p_{\xi}
  \]
  が単射であるとき、$n$次元\textbf{統計的モデル}、または、\textbf{パラメトリックモデル}、または単に\textbf{モデル}という。
\end{definition}
測度空間$(\Omega,\mathcal{B})$と、$n$個の実数でパラメトライズされた確率密度の族を統計的モデルというわけです。
Bourbaki流に$(\Omega,\mathcal{B},\Xi,S)$とでも書くべきでしょうか。

次は正規分布が統計的モデルの定義をみたしていることを見ます。
\begin{example}
  平均$\mu\in\mathbb{R}$、偏差$\sigma>0$の正規分布
  \[
    p(x;\mu,\sigma)=\frac{1}{\sqrt{2\pi}\sigma}\exp\left\{-\frac{(x-\mu)^2}{2\sigma^2}\right\}
  \]
  の族は、2次元統計的モデルである。
\end{example}

上記の具体例を座右に、Fisher情報量を定義しましょう。
\begin{definition}[Fisher情報量]
  $\Xi\subset\mathbb{R}^n$は開集合とし、$S$を測度空間$(\Omega,\mathcal{B})$の$\Xi$上の$n$次元統計的モデルとする。
  また各$x\in\Omega$に対して、写像
  \[
    \Xi\to\mathbb{R};\xi\mapsto p(x;\xi)
  \]
  は滑らかであるとする。
  このとき、以下の期待値からなる行列
  \[
    g_{ij}(\xi):=E_\xi\left[\frac{\partial \log p(x;\xi)}{\partial \xi^i}\frac{\partial \log p(x;\xi)}{\partial \xi^j}\right]
  \]
  を\textbf{Fisher情報量}あるいは\textbf{Fisher情報行列}と呼ぶ。
\end{definition}

\begin{example}[正規分布のFisher情報量]
  平均$\mu\in\mathbb{R}$、偏差$\sigma>0$の正規分布の場合
  \[
    \log p(x;\mu,\sigma)=-\frac12\log 2\pi - \log \sigma - \frac{(x-\mu)^2}{2\sigma^2}
  \]
  より
  \begin{align*}
    \frac{\partial \log p(x;\mu,\sigma)}{\partial \mu}&=\frac{x-\mu}{\sigma^2}\\
    \frac{\partial \log p(x;\mu,\sigma)}{\partial \sigma}&= - \frac1\sigma + \frac{(x-\mu)^2}{\sigma^3}
  \end{align*}
  期待値を計算すると、
  \begin{align*}
    g_{\mu\mu}&=\int_\mathbb{R}\frac{(x-\mu)^2}{\sigma^4}\frac{1}{\sqrt{2\pi}\sigma}\exp\left\{-\frac{(x-\mu)^2}{2\sigma^2}\right\}dx=\frac1{\sigma^2}\\
    g_{\mu\sigma}=g_{\sigma\mu}&=\int_\mathbb{R}\frac{x-\mu}{\sigma^2}\left\{- \frac1\sigma + \frac{(x-\mu)^2}{\sigma^3}\right\}\frac{1}{\sqrt{2\pi}\sigma}\exp\left\{-\frac{(x-\mu)^2}{2\sigma^2}\right\}dx=0\\
    g_{\sigma\sigma}&=\int_\mathbb{R}\left\{- \frac1\sigma + \frac{(x-\mu)^2}{\sigma^3}\right\}^2\frac{1}{\sqrt{2\pi}\sigma}\exp\left\{-\frac{(x-\mu)^2}{2\sigma^2}\right\}dx=\frac{2}{\sigma^2}
  \end{align*}
  となっている。
  計量としてまとめると、
  \[
    ds^2=\frac{d\mu^2+2d\sigma^2}{\sigma^2}
  \]
  となり、これは上半平面$\{(\mu,\sigma)\mid\mu\in\mathbb{R}, \sigma>0\}$におけるPoincar\'e計量によく似た計量である。
\end{example}



\section{$\alpha$接続}

\begin{comment}
  TODO
  具体的には、まず\textbf{甘利-Chentsovテンソル}を
  \[
    C_{ijk}:=E_\xi\left[\frac{\partial \log p(x;\xi)}{\partial \xi^i}\frac{\partial \log p(x;\xi)}{\partial \xi^j}\frac{\partial \log p(x;\xi)}{\partial \xi^k}\right]
  \]
  で定義することができます。
  次に$\nabla^{(0)}$を、Fisher計量に付随するRiemann計量とします。
  これらを用いて、実数$\alpha$とベクトル場$X,Y,Z$に対して
  \[
    \langle\nabla_X^{(\alpha)}Y,Z\rangle=\langle\nabla_X^{(0)}Y,Z\rangle-\frac{\alpha}{2}C(X,Y,Z)
  \]
  を満たすアフィン接続$\nabla^{(\alpha)}$を、\textbf{$\alpha$接続}といいます。
  $\alpha$接続の自然さは、$-\alpha$が双対接続になっているということです。
  すなわち、
  \[
    Xg(Y,Z)=g(\nabla_X^{(\alpha)}Y,Z)+g(Y,\nabla_X^{(-\alpha)}Z)
  \]
  を満たします。
  \end{comment}

以下、例の草案

正規分布(一次元ガウス分布族)
ポアソン分布
$\Omega$が有限集合の場合

sgc
離散分布族
一般の統計モデル
正測度空間
行列の空間
神経回路網の空間

  \part{統計学の基礎}
  \chapter{統計学の目指すところ}

統計学とは何を目的としているのか?という点について、僕の手元にある教科書ではあまりサクッと触れられているものはありませんでした。
僕は統計学の基礎を、情報幾何学に向けて理解していきたいと思っているのですが、そこに現れる種々の「統計学的に自然な概念」がどうして自然なのかがわからなくて難儀しています。
このパートは、そのような人(主に僕)のために、明確な目的意識を作るために、全体を俯瞰しようと思います。

前提知識としては、ふんわりと測度論的確率論がわかっていて、確率変数の平均値、期待値を計算できることを前提とします。

\section{統計学の目的}

統計学の目標を一言でいうと、「乱雑に集められたデータから、真の特徴を発見し、産業や医療、実験科学の役に立て、もって未来の予測を行うこと」のようです。
自分なりに統計学の目的を分類してみると、次のような目標があるようです。
\begin{itemize}
  \item (帰納的)推論
  \item 「ノイズ」と「シグナル」の分離
  \item 意思決定とリスク管理(結局どう統計データを扱うのか?
\end{itemize}

乱雑に集められたデータというのは、例えばどういうものがあるかというと、
\begin{description}
  \item[実験の分野] 真の法則を発見するために、ある実験結果の膨大なデータ。
  \item[天体観測] 観測データのズレがどれだけ大きいか\footnote{
    ガウスが研究していたようです。
    この観測誤差が正規分布になっていることをガウスは見つけたようです。
  }。
  \item[身体測定] 国民全員の身長の測定データ。
  \item[選挙] 出口調査。
\end{description}

後述するノイズの大きさはあれど、これらはそのまま記録してデータとして積み上げておく必要があります。



\section{推定}

母集団をすべて調べるというのは、いくつかの面で困難です。
例えば
\begin{itemize}
  \item 人件費やコストの問題
  \item 母集団がそもそも無限個存在している
\end{itemize}
などの理由があります。

その時は母集団からいくつかサンプル(\textbf{標本})を観察し、母集団の傾向をつかむという\textbf{推定}を行います。
例えば国民全員の身長を調べるのは大変なので、ランダムに2000人を選んで身長を測定し、その平均値を国民の身長の代表値として用いることができます。
これを\textbf{標本平均}と言いますが、この標本平均は母平均、すなわち本当の国民全員の身長の平均とは一致しないでしょう。
しかしこれが母平均とどれだけズレているかは、標本平均の分散を考えることで確率的にわかるものなのです。

推定に使われるデータは以下のようなものがあります。
\begin{center}
  \textbf{平均、分散、標準偏差、四分位偏差、中央値、モード(最頻値)、共分散、相関係数、etc.}
\end{center}
これらを\textbf{代表値}といいます。

代表値の中でも、平均と分散(標準偏差)は格別と言えるでしょう。
その根底には、確率論における強力な定理が存在します。
\begin{description}
  \item[\textbf{大数の法則}] サンプル数を増やせば標本平均は真の平均に(確率収束、あるいは概収束の意味で)近づいていく
  \item[\textbf{中心極限定理}] 母集団がどのような分布であっても(分散が有限であれば)、サンプルサイズが十分大きければ、その「標本平均の分布」は正規分布に近づく
\end{description}
これらにより、母集団の分布詳細が不明でも、標本平均の振る舞いに関しては正規分布を用いた解析が正当化されるのです\footnote{
  実は、正規分布を含む扱いやすい確率分布(\textbf{指数型分布族})においては、標本平均さえ知っていれば、元のデータが持っていたパラメータ推定に関する情報を一切失わない(\textbf{十分統計量}である)ことも知られています。
  この「情報のロスがない」という性質も、平均値が自然な概念として扱われる理由の一つです。
}。
大数の法則および中心極限定理は、測度論的確率論をベースとして証明されるれっきとした定理です。

このように意味では、推定という作業は、無数の可能性(確率分布の空間)の中から、手元のデータを最もよく説明できる「一点(特定のパラメータを持つ確率分布)」を探し出す作業と言い換えることができます。
これは幾何学において、曲面上の点を探す行為に似ています。



\section{ノイズとシグナル}

一般的に中心極限定理より、サンプルが多ければ多いほど、分散が小さくなり、標本平均は母平均に近いと言えます。
しかし「多く」といっても、何件くらいの標本を調べれば満足いく精度の正規分布が分かるかは不確実性があります。
この不確実性を、確率論の力を使って「99\%正しい」とか、「対立する仮定(帰無仮説)は棄却できる」などと断言できるようになります。
このように、標本平均の分散を考えることで、「ノイズ」と「シグナル」を分離することができます。
例えば、
\begin{description}
  \item[実験の分野] 真の法則を発見するために、ある実験結果の膨大なデータ。\\
    $\to$ 結果を人間が目盛りを読んでノートに記述するノイズ、気温や湿度の違い、等。
  \item[天体観測] 観測データのズレがどれだけ大きいか。\\
    $\to$ 同上
  \item[身体測定] 国民全員の身長の測定データ。\\
    $\to$ 測定機器の設備の違い、地域ごとの食文化の違いなど。
  \item[選挙] 出口調査\\
    $\to$ 記憶違い。嘘をつかれる。
\end{description}

ノイズとシグナルを分ける有名な方法としては、\textbf{カイ二乗分布}を使用することが挙げられます。
例えばある実験を30回繰り返すと、その標本の分散が、真の分散と約2倍の誤差の範囲に収まることを99\%保証されます。
逆に言えば、分散が2倍以上離れてしまう可能性は1\%ぐらいに収まります。
実験レポートではここまで計算・分析・考察して報告しなければなりません。



\section{意思決定とリスク管理}

この部分が、統計学の最も役にたつ部分です。
\begin{description}
  \item[実験の分野] 真の法則を発見するために、ある実験結果の膨大なデータ。\\
    $\to$ 結果を人間が目盛りを読んでノートに記述するノイズ、気温や湿度の違い、等。\\
    $\to$ 統計分析によって、ノイズが混ざっている可能性は1\%未満であると言える。
  \item[天体観測] 観測データのズレがどれだけ大きいか。\\
    $\to$ 同上
  \item[身体測定] 国民全員の身長の測定データ。\\
    $\to$ 測定機器の設備の違い、地域ごとの食文化の違いなど。\\
    $\to$ 建築において、ドアの高さの規格が作れる。優位に身長が低い地域があれば、何か健康被害があるのではないかと調査する必要性が生まれる。等。\\
  \item[選挙] 出口調査\\
    $\to$ 記憶違い。嘘をつかれる。\\
    $\to$ 上記のノイズを排除しても20時ちょうどに当選確実の報を出せる。
\end{description}
より具体的には、実験ノイズを排除することで、より普遍的な自然法則に近づくことができるでしょう。
あるいは日本人の身長の中央値を知ることができれば、その中央値よりも少し高いドアを作れば大体良いでしょう。
このことによって、製造メーカーはドアの規格として高さ2mぐらいのドアを作れば、コストに見合う大量生産を行い、大きな利益を得るでしょう。

医療の分野では、特に人の生死がかかわっているので、手術の成功率を統計的手法によって厳密に計算しなければならないです(\textbf{医療統計})。
そして「それでも手術を受けますか?」という提案を医者側からしなければなりません。
このことからも実に重い責務を統計学は担っていることがわかるでしょう。

このように、「真の母集団と標本平均のズレ」がどれだけの確率で発生するかを理論的に検証し、もって産業や医療、あるいは他学問に役立てることができるのです\footnote{
  昔の番組のトリビアの泉では、市場調査で2000人のサンプルを調べれば十分信頼できる結果が得られるというくだりが何度もありました。
}。
統計学では、このような検定方法の理論的裏付けを与えるのです。

特に近年では、ビッグデータ・深層学習・LLMの時代に突入してきており、これまで理論的には無茶とされていたことが覆されてきています。
つまり旧来、深層学習などの分野における素朴な勾配法では、過学習などの問題があり使い物にならないといわれていましたが、この定説を打ち破り、「やってみればうまくいってしまった」という状況なのです\cite{Am25}。
であれば、「なぜうまくいってしまったのか?」という理論的根拠を与えるために、情報幾何学や統計学で発展が急速に進んでいます。
また統計学が情報幾何学を通じて幾何学と結んでしまったため、機械学習と幾何学、量子論、はてはゲージ理論・素粒子論、代数幾何学と結びつき、さらに深い理解が求められるようになってきています。



\section{まとめ}

統計学には以下の大目標があります。
\begin{itemize}
  \item (帰納的)推論
  \item 「ノイズ」と「シグナル」の分離
  \item 意思決定とリスク管理(結局どう統計データを扱うのか?
\end{itemize}

これらを研究することで、産業・医療、はては物理や純粋数学において大いに役立つ学問であることが想像できたでしょうか?
僕たちは、これらを主軸において、統計学の勉強を進めることにします。

  \chapter{代表値}

代表値には、ざっと以下のものがあります。
\begin{itemize}
  \item 中心を意味する代表値
    \begin{itemize}
      \item (算術)平均値
      \item 中央値
      \item モード(最頻値)
    \end{itemize}
  \item モーメントを意味する代表値
    \begin{itemize}
      \item 2次のモーメント:分散 $\to$ 標準偏差
      \item 3次のモーメント:歪度(分布の非対称性)
      \item 4次のモーメント:尖度(裾の厚さ、頂点の鋭さ)
    \end{itemize}
  \item 多変量解析における代表値
    \begin{itemize}
      \item 共分散・相関係数
      \item 多変量共分散
    \end{itemize}
\end{itemize}

情報幾何学ではとりわけ、平均値と分散、3次のモーメントが重要になります。
平均値はさることながら、分散という2次のモーメントと、3次のモーメントは、「適切な確率密度を探す」という推定において、それぞれ1次の近似と2次の近似を与えます。
特に多変量解析において、分散は共分散という半正定値行列をつくり、\textbf{Fisher情報行列}というRiemann計量と深いつながりがあります。
その文脈で3次のモーメントは、\textbf{KLダイバージェンス}という二つの確率密度の距離(のようなもの)を定義し、実際これは接空間上の接続と深いかかわりがあります。
従ってこの章では、平均・分散・3次のモーメントについて、1次元の場合に勉強していきましょう。



\section{平均値}

\begin{definition}
  $(\Omega,\mathcal{F},P)$を確率空間とし、$X:\Omega\to\mathbb{R}$を確率変数とする。
  このとき、
  \[
    \mu:=E[X]=\int_\Omega X(\omega)dP(\omega)
  \]
  を、確率変数$X$の\textbf{平均値}あるいは\textbf{期待値}という。
\end{definition}

\begin{example}[サイコロの平均値]
  \begin{align*}
    \Omega&:=\{1,2,3,4,5,6\}\\
    \mathcal{F}&:=2^\Omega\\
    P(A)&=\sum_{\omega\in A}\frac16
  \end{align*}
  とすると、$(\Omega,\mathcal{F},P)$は確率空間となる。
  \[
    X(\omega):=\omega
  \]
  とすれば、$X:\Omega\to\mathbb{R}$は確率変数である。
  このとき、$X$の平均値は
  \[
    E[X]=\int_\Omega X(\omega)dP(\omega)=\sum_{\omega=1}^6 \omega \frac16 =\frac{21}{6}=\frac72=3.5
  \]
\end{example}

平均値は統計処理として最も多く使われるものですが、極端な外れ値に引っ張られる傾向にあるので、時と場合によって使い分けなければなりません。
貯蓄額を例に取ると、1992年の統計では平均551万円であり、中央値が489万円だったといいます\cite{Ina-Yama-Yos92}\footnote{
  今(2026/01/01)の感覚としてはなかなか高収入ですね。。。
}。
中央値は庶民の感覚に大体合致していそうですが、平均はごく一部の大富豪に引っ張られてだいぶ高く出てしまっています。
安易に平均・期待値に過大な期待をしてしまうと、ギャンブルで大損するかもしれませんね。



\section{分散}

\begin{definition}
  $(\Omega,\mathcal{F},P)$を確率空間とし、$X:\Omega\to\mathbb{R}$を確率変数、$\mu:=E[X]$を$X$の平均値とする。
  このとき、
  \[
    V(X):=E[(X-\mu)^2]=\int_\Omega \{X(\omega)-\mu\}^2 dP(\omega)
  \]
  を、確率変数$X$の\textbf{分散}という。$\sigma(X)^2:=V(X)$とも書く。
\end{definition}

\begin{definition}
  $(\Omega,\mathcal{F},P)$を確率空間とし、$X:\Omega\to\mathbb{R}$を確率変数、$V(X)$を$X$の分散とする。
  このとき、
  \[
    \sigma(X):=\sqrt{V(X)}
  \]
  を、確率変数$X$の\textbf{標準偏差}という。
\end{definition}

定義をよく見ればわかる通り、分散は、$X$の実際の値が平均からどれぐらいズレうるかを意味しています。
つまり、$X$の値が平均値から離れれば離れるほど、分散は大きくでてきます。
二つの具体例を見てみましょう。

\begin{example}[公平なサイコロの分散・標準偏差]
  \begin{align*}
    \Omega&:=\{1,2,3,4,5,6\}\\
    \mathcal{F}&:=2^\Omega\\
    P(A)&=\sum_{\omega\in A}\frac16
  \end{align*}
  とすると、$(\Omega,\mathcal{F},P)$は確率空間となる。
  \[
    X(\omega):=\omega
  \]
  とすれば、$X:\Omega\to\mathbb{R}$は確率変数である。
  このとき、$X$の分散は
  \[
    V(X)=\int_\Omega \{X(\omega)-3.5\}^2 dP(\omega)=\sum_{\omega=1}^6 (\omega-3.5)^2 \frac16 =\frac{35}{12}=2.91666\cdots
  \]
  標準偏差は
  \[
    \sigma(X)=\sqrt{35/12}=1.708\cdots
  \]
\end{example}

この場合は$X$が平均$3.5$から、せいぜい標準偏差$\pm1.708$ぐらいバラけていることがわかります。
これは\textbf{平均値$3.5$あたりが出ると信じてサイコロを投げたときのリスク感}と言えます。
このサイコロは公平なので、$3$や$4$あたりに賭けておけば、損しても得しても、標準偏差$1.7$ぐらいの誤差に収まる、というイメージです。

では、偏ったサイコロの場合はどうでしょう?

\begin{example}[偏ったサイコロの分散・標準偏差]
  \begin{align*}
    \Omega&:=\{1,2,3,4,5,6\}\\
    \mathcal{F}&:=2^\Omega\\
    P(A)&=\begin{cases}
      1 & (A=\{1,6\})\\
      0.5 & (A=\{1\} \text{ or } \{2\})\\
      0 & (\text{otherwise})
    \end{cases}
  \end{align*}
  とすると、$(\Omega,\mathcal{F},P)$は確率空間となる。
  \[
    X(\omega):=\omega
  \]
  とすれば、$X:\Omega\to\mathbb{R}$は確率変数である。
  このとき、$X$の分散は
  \[
    V(X)=\int_\Omega \{X(\omega)-3.5\}^2 dP(\omega)= \frac{2.5^2+2.5^2}{2} = 6.25
  \]
  標準偏差は
  \[
    \sigma(X)=\sqrt{6.25}=2.5
  \]
\end{example}

これはまずいですね。
平均値が$3.5$であることは一緒なのですが、それを期待して$3$や$4$に賭けても絶対当たらないので、公平なサイコロに比べて\textbf{平均値を信じたときの損得のブレ幅が大きいです}。
なので$1$か$6$に賭けざるを得ないのですが、そうするならもうコイン投げるのと同じことです。
標準偏差の感覚がわかってきたでしょうか?

最後に簡単な定理を紹介します。
計算上ちょっと便利になります。

\begin{theorem}
  $(\Omega,\mathcal{F},P)$を確率空間とし、$X:\Omega\to\mathbb{R}$を確率変数とする。
  $E[X^2]<+\infty$であるとすると、以下が成り立つ。
  \[
    V[X]=E[X^2]-E[X]^2
  \]
\end{theorem}
\begin{proof}
  TODO: chap70.tex以降
\end{proof}



\section{$k$次モーメント}

\begin{definition}
  $(\Omega,\mathcal{F},P)$を確率空間とし、$X:\Omega\to\mathbb{R}$を確率変数、$\mu:=E[X]$を$X$の平均値とする。
  $k>0$を整数とするとき、
  \[
    \mu_k(X):=E[(X-\mu)^k]=\int_\Omega \{X(\omega)-\mu\}^k dP(\omega)
  \]
  を、確率変数$X$の\textbf{$k$次モーメント}という。
\end{definition}

$1$次モーメントは
\[
  \mu_1(X)=E[X-\mu]=E[X]-\mu=0
\]
なのであまり意味がないですが、一応入れておきました。
とはいえ$\mu_1(X)=0$という事実は、$X$の分布それ自体が平均値を\textbf{重心としている}という意味を汲み取れます。

$2$次モーメントは
\[
  \mu_2(X)=E[(X-\mu)^2]
\]
なので、分散に他なりません。

新概念は$k\ge3$のときになります。

\subsection{歪度}

\begin{definition}
  $(\Omega,\mathcal{F},P)$を確率空間とし、$X:\Omega\to\mathbb{R}$を確率変数、$\sigma(X)$を$X$の標準偏差とする。
  このとき
  \[
    g_1(X):=\frac{\mu_3(X)}{\sigma(X)^3}
  \]
  を、確率変数$X$の\textbf{歪度}(\textit{skewness})という。
\end{definition}

$3$次のモーメントは、奇数乗になったことで分散で打ち消したプラスマイナスが復活しました。
そのため、もし分布が左右対称なら、プラスとマイナスが完全に打ち消し合って$\mu_3(X)=0$になります。
逆に言えば、値が残る場合、それは「\textbf{どちらかの裾が長く伸びている}」ことを示唆します。
それで「歪度」というわけです。

\begin{example}[公平なサイコロの$3$次モーメント・歪度]
  \begin{align*}
    \Omega&:=\{1,2,3,4,5,6\}\\
    \mathcal{F}&:=2^\Omega\\
    P(A)&=\sum_{\omega\in A}\frac16
  \end{align*}
  とすると、$(\Omega,\mathcal{F},P)$は確率空間となる。
  \[
    X(\omega):=\omega
  \]
  とすれば、$X:\Omega\to\mathbb{R}$は確率変数である。
  このとき、$X$の$3$次モーメントは
  \[
    \mu_3(X)=\int_\Omega \{X(\omega)-3.5\}^3 dP(\omega)=\sum_{\omega=1}^6 (\omega-3.5)^3 \frac16 = 0
  \]
  ゆえに歪度は
  \[
    g_1(X)=0
  \]
\end{example}

分散があり、歪度0というのは、平均値に綺麗に値が固まっていることを意味しています。
もし歪度が0でない場合は、平均ではないところにピークがあるデータを意味しており、その意味で「ゆがんだデータ」と言えましょう。

\subsection{尖度}

\begin{definition}
  $(\Omega,\mathcal{F},P)$を確率空間とし、$X:\Omega\to\mathbb{R}$を確率変数、$\sigma(X)$を$X$の標準偏差とする。
  このとき
  \[
    g_2(X):=\frac{m_4(X)}{\sigma(X)^4}
  \]
  を、確率変数$X$の\textbf{尖度}(\textit{kurtosis})という。
\end{definition}

偶数乗なので再び符号は消えますが、2乗よりもさらに強烈に「平均から遠い値」を増幅します。

\begin{example}[公平なサイコロの$4$次モーメント・尖度]
  \begin{align*}
    \Omega&:=\{1,2,3,4,5,6\}\\
    \mathcal{F}&:=2^\Omega\\
    P(A)&=\sum_{\omega\in A}\frac16
  \end{align*}
  とすると、$(\Omega,\mathcal{F},P)$は確率空間となる。
  \[
    X(\omega):=\omega
  \]
  とすれば、$X:\Omega\to\mathbb{R}$は確率変数である。
  このとき、$X$の$4$次モーメントは
  \[
    \mu_4(X)=\int_\Omega \{X(\omega)-3.5\}^4 dP(\omega)=\sum_{\omega=1}^6 (\omega-3.5)^4 \frac16 = \frac{707}{48} = 14.7291666\cdots
  \]
  ゆえに尖度は
  \[
    g_2(X)=\frac{12^2}{35^2}\frac{707}{48}=1.7314\cdots
  \]
\end{example}

\begin{example}[偏ったサイコロの分散・標準偏差]
  \begin{align*}
    \Omega&:=\{1,2,3,4,5,6\}\\
    \mathcal{F}&:=2^\Omega\\
    P(A)&=\begin{cases}
      1 & (A=\{1,6\})\\
      0.5 & (A=\{1\} \text{ or } \{2\})\\
      0 & (\text{otherwise})
    \end{cases}
  \end{align*}
  とすると、$(\Omega,\mathcal{F},P)$は確率空間となる。
  \[
    X(\omega):=\omega
  \]
  とすれば、$X:\Omega\to\mathbb{R}$は確率変数である。
  このとき、$X$の$4$次モーメントは
  \[
    \mu_4(X)=\int_\Omega \{X(\omega)-3.5\}^4 dP(\omega)= \frac{2.5^4+2.5^4}{2} = 39.0625
  \]
  ゆえに尖度は
  \[
    g_2(X)=\frac{39.0625}{6.25^2}=1
  \]
\end{example}

尖度はその言葉が示す通り「確率分布の尖り具合」を意味しています。
比較対象として、標準正規分布の尖度を計算すると、

\begin{align*}
  m_4&=\int_\mathbb{R} x^4\exp\left(-\frac{x^2}{2}\right) dx = 3
\end{align*}
ゆえに
\[
  g_2=3
\]
となり、標準正規分布に比べて偏ったサイコロの尖度は低いと見ることができます。
値が両端($1$と$6$)に完全に分かれているため、中央がスカスカで、分布としては最も「尖っていない」状態と言えます。

  \chapter{母集団・標本}

統計学には以下の大目標があります。
\begin{itemize}
  \item (帰納的)推論
  \item 「ノイズ」と「シグナル」の分離
  \item 意思決定とリスク管理(結局どう統計データを扱うのか?
\end{itemize}

前章では推論を行うための道具を、一つの確率変数に対していくつか紹介しました。
しかし、例えば実験誤差をどう「分離」し、結果の妥当性を判断するかという統計学の役割を果たすためには、一つの確率変数を見ているだけでは不可能です。
実験を一回だけ行って「うまくいきました!」と報告しても、その結果が様々なノイズによって偶然そうなった可能性があるため、当然リジェクトされるでしょう。
残念なことに歴史上そのようなフィルターが機能しなかったことも多くあり、後世に甚大な影響を及ぼした例は枚挙にいとまがありません。
高温超電導物質捏造事件や旧石器捏造事件、またSTAP細胞事件など。
この背景には、そういった統計学的リテラシーの低さがあると思われます(個人の感想です)。

もとい。
そういった不誠実な事件を減らすためにも、統計学はかなり中立的な方法を提示してくれています。
その土台となる数学的な定義集をこの章でも行っていきましょう。



\section{母集団}

前章では平均、分散・標準偏差、歪度、尖度を紹介しましたが、統計学においてそもそも何のためにそれらを計算するのか、という話をしていませんでした。
いや、前章でさらっと話はしていますが、測度論の言葉を使ってしっかり明確に定義していきましょう。

\begin{definition}
  確率空間$(\Omega,\mathcal{F},P)$を\textbf{母集団}と呼ぶ。
  また$\Omega$上の確率変数$X$を一つ固定するとき、その確率分布(像測度)
  \[
    \mu_X:=P\circ X^{-1}:\mathcal{B}(\mathbb{R})\to[0,1]
  \]
  を\textbf{母分布}と呼ぶ。
\end{definition}



\section{独立・従属}

サイコロを2回投げることを考えると、1回目の試行の結果は2回目の試行に影響を及ぼさないと考えることができます。
実験においても、1回目の測定によって2回目の測定に影響が及ばないよう細心の注意を払うべきでしょう\footnote{
  ここは性善説をとらざるをえないでしょう。
}。
例えば化学実験をするたびにフラスコを純水で洗う等です。
統計学では、このような\textbf{「互いに無関係である」という状況を数学的に定式化するために}、以下のような定義をします。

\begin{definition}
  $(\Omega,\mathcal{F},P)$を母集団、$X_1,X_2$を$\Omega$上の確率変数とする。
  任意のボレル集合$B_1,B_2\in\mathcal{B}(\mathbb{R})$に対して、
  \[
    P(X_1^{-1}(B_1)\cap X_2^{-1}(B_2))=P(X_1^{-1}(B_1))P(X_2^{-1}(B_2))
  \]
  を満たす時、$X_1$と$X_2$は互いに\textbf{独立}であるという。
  $X_1$と$X_2$が独立でない時、互いに\textbf{従属}しているという。
\end{definition}
$P(X_1^{-1}(B_1)\cap X_2^{-1}(B_2))=P(X_1^{-1}(B_1))P(X_2^{-1}(B_2))$は、慣例的な表記では
\[
  P(X_1 \in B_1, X_2 \in B_2) = P(X_1 \in B_1)P(X_2 \in B_2)
\]
と記述します。

例えば「国民」という母集団に対して、「身長」という確率変数$X$と、「体重」という確率変数$Y$を考えましょう。
直感的には、身長が高ければ高いほど体重も大きくなる(身長に関する情報が、体重の予測に影響を与える)という感覚があります。
これは従属の例です。



\section{共分散・相関係数}

TODO

\subsection{共分散行列}

TODO



\section{標本}

母集団から何個かデータを取ってきたものを\textbf{標本}といいます。
数学的には、母集団分布(母分布)と同じ分布に従う、独立な確率変数の列として定義されます。

\begin{definition}
  母集団$(\Omega,\mathcal{F},P)$上の確率変数$X$の分布(母分布)を$\mu$とする。
  このとき、すべて母分布$\mu$に従い、かつ互いに独立な$n$個の確率変数の組
  \[
  X_1, X_2, \dots, X_n
  \]
  を、大きさ$n$の\textbf{無作為標本}(あるいは単に\textbf{標本})と呼ぶ。
\end{definition}

実験の例では、$n$回繰り返して得られるデータ全体の組 $(X_1, \dots, X_n)$ が標本です。
個々の $X_i$ は $i$ 回目の測定における値を意味します。

国民という母集団の例では、無作為に選ばれた$n$人のアンケート結果の集まりが標本となります。
解答用紙に書かれた「身長」などの一つの数値が、それぞれの $X_i$ に対応します。

\subsection{iid}

TODO



\section{統計量}

大きさ$n$の標本$X_1,\dots,X_n$から演算して得られる一つの確率変数を\textbf{統計量}といいます。
統計学の目標とするべき代表的な統計量は以下のようなものがあります。

\begin{definition}
  $X_1,\dots,X_n$を母集団$(\Omega,\mathcal{F},P)$の標本とする。
  このとき、確率変数
  \[
    \overline{X}_n:=\frac1n\sum_{i=1}^nX_n
  \]
  を、$X_1,\dots,X_n$の\textbf{標本平均}という。

  \[
    \overline{S}_n^2:=\frac1n\sum_{i=1}^n(X_i-\overline{X}_n)
  \]
  を、$X_1,\dots,X_n$の\textbf{標本分散}という。
\end{definition}

\subsection{国民の身長の例}

$\Omega$は国民の集合で、$\omega\in\Omega$は国民ひとりひとりです。
このとき確率測度として取るべき$P$は、数え上げ測度の正規化
\[
  P(A):=\frac{\#A}{\#\Omega}
\]
が自然でしょう。
また$X$は国民$\omega$の身長を意味し、母分布の値
\[
  \mu_X([150,160])=P(150\le X\le 160)
\]
は、身長150cmから160cmであるような国民の割合を意味します。

\subsection{実験の測定誤差}

実験では、期待する「真の値」というものがあります。
これを$\mu\in\mathbb{R}$とします。

$\Omega$は、ありとあらゆる可能性を含んだ、測定の瞬間の状態の集合です。
$\omega\in\Omega$は、「ある一回の具体的な測定施行」そのものになります。
$\omega$には、その測定を行った時のノイズ(気温・気圧、手振れ等)を含んでいると考えます。

確率変数$X$は、観測$\omega$に対して観測機器に表示される値を意味します。
$X$の母分布$\mu_X$は、おそらく期待する「真の値」$\mu$をピークとする確率密度関数を持つでしょう。

実際、もし個々の実験が独立かつ同じ条件(独立同分布)であれば、\textbf{大数の法則}によって実験を繰り返せば繰り返すほどその標本平均は「真の値 $\mu$」に近づきます。
また、実験回数$n$が十分大きければ、元の母分布$\mu_X$がどのような形であっても、\textbf{中心極限定理}によって「標本平均 $\overline{X}_n$ のばらつき(真の値からの誤差)」は正規分布で近似できることが保証されます。

実験レポートでは 実験回数$n$、平均 $\overline{X}_n$、不偏分散 $s^2$を報告し、推定精度の指標(標準誤差など)を示すべきでしょう。


  \part{確率測度論}
  \chapter{測度論}

現代確率論は、ルベーグ積分論に起源をもつ(コルモゴロフによる)\textbf{公理的確率論}で基礎づけされています。

\section{可測空間}

\section{確率空間}

\section{確率変数}

\section{平均・期待値}

\section{分散・標準偏差}

\section{モーメント}
  \printbibliography
\end{document}
