\chapter*{前提知識一覧}

\begin{description}
  \item[ガウス曲面論] 多様体以前に、その動機となったガウスの曲面論を知っておくのは教養として良いことです。
  \begin{itemize}
    \item Kobayashi, \textit{Differential geometry of curves and surfaces} \cite{Ko19}\\
    (なぜか日本語版がCiNiiで見つからなかったよ。日本語版もあるから安心してね。)
    \item Singer \& Thorpe (翻訳: 松江, 一楽), \textbf{トポロジーと幾何学入門}\cite{Sin-Tho76} (第7,8章)
  \end{itemize}
  \item[多様体論] 以下の本は鉄板ですね。
  なにも言いますまい。四の五の言わずに全部読んでください。
  \begin{itemize}
    \item 松島, \textbf{多様体入門} \cite{Matsushima17}
    \item 松村, \textbf{多様体の基礎} \cite{Matsumoto88}
  \end{itemize}
  \item[代数トポロジー] 多様体論を経た学生には是非Bott\&Tuをおすすめします。
  Hatcherは英語ですが分かりやすく、無料でPDFが配布されているため非常におすすめです。
  このあたりから普通に英語が読めるようになっておいた方がいいです。
  \begin{itemize}
    \item Bott \& Tu, \textit{Differential forms in algebraic topology} \cite{BT82}
    \item Hatcher, \textit{Algebraic topology} \cite{Hat01}
  \end{itemize}
  \item[Riemann幾何] 上記の書籍にもある程度Riemann幾何が解説されていますが、辞書的な本をさらっと読んでおくといいでしょう。
  Riemann幾何は相対論と両輪で理解するとはかどりますので、一冊紹介しましょう。
  \begin{itemize}
    \item 酒井, \textit{リーマン幾何学} \cite{Sak92}
    \item Schutz, (翻訳: 江里口, 二間瀬), \textit{シュッツ相対論入門} \cite{Sch23}
  \end{itemize}
\end{description}
