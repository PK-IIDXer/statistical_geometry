\chapter{統計学の目指すところ}

統計学とは何を目的としているのか?という点について、僕の手元にある教科書ではあまりサクッと触れられているものはありませんでした。
僕は統計学の基礎を、情報幾何学に向けて理解していきたいと思っているのですが、そこに現れる種々の「統計学的に自然な概念」がどうして自然なのかがわからなくて難儀しています。
このパートは、そのような人(主に僕)のために、明確な目的意識を作るために、全体を俯瞰しようと思います。

前提知識としては、ふんわりと測度論的確率論がわかっていて、確率変数の平均値、期待値を計算できることを前提とします。

\section{統計学の目的}

統計学の目標を一言でいうと、「乱雑に集められたデータから、真の特徴を発見し、産業や医療、実験科学の役に立て、もって未来の予測を行うこと」のようです。
自分なりに統計学の目的を分類してみると、次のような目標があるようです。
\begin{itemize}
  \item (帰納的)推論
  \item 「ノイズ」と「シグナル」の分離
  \item 意思決定とリスク管理(結局どう統計データを扱うのか?
\end{itemize}

乱雑に集められたデータというのは、例えばどういうものがあるかというと、
\begin{description}
  \item[実験の分野] 真の法則を発見するために、ある実験結果の膨大なデータ。
  \item[天体観測] 観測データのズレがどれだけ大きいか\footnote{
    ガウスが研究していたようです。
    この観測誤差が正規分布になっていることをガウスは見つけたようです。
  }。
  \item[身体測定] 国民全員の身長の測定データ。
  \item[選挙] 出口調査。
\end{description}

後述するノイズの大きさはあれど、これらはそのまま記録してデータとして積み上げておく必要があります。



\section{推定}

母集団をすべて調べるというのは、いくつかの面で困難です。
例えば
\begin{itemize}
  \item 人件費やコストの問題
  \item 母集団がそもそも無限個存在している
\end{itemize}
などの理由があります。

その時は母集団からいくつかサンプル(\textbf{標本})を観察し、母集団の傾向をつかむという\textbf{推定}を行います。
例えば国民全員の身長を調べるのは大変なので、ランダムに2000人を選んで身長を測定し、その平均値を国民の身長の代表値として用いることができます。
これを\textbf{標本平均}と言いますが、この標本平均は母平均、すなわち本当の国民全員の身長の平均とは一致しないでしょう。
しかしこれが母平均とどれだけズレているかは、標本平均の分散を考えることで確率的にわかるものなのです。

推定に使われるデータは以下のようなものがあります。
\begin{center}
  \textbf{平均、分散、標準偏差、四分位偏差、中央値、モード(最頻値)、共分散、相関係数、etc.}
\end{center}
これらを\textbf{代表値}といいます。

代表値の中でも、平均と分散(標準偏差)は格別と言えるでしょう。
その根底には、確率論における強力な定理が存在します。
\begin{description}
  \item[\textbf{大数の法則}] サンプル数を増やせば標本平均は真の平均に(確率収束、あるいは概収束の意味で)近づいていく
  \item[\textbf{中心極限定理}] 母集団がどのような分布であっても(分散が有限であれば)、サンプルサイズが十分大きければ、その「標本平均の分布」は正規分布に近づく
\end{description}
これらにより、母集団の分布詳細が不明でも、標本平均の振る舞いに関しては正規分布を用いた解析が正当化されるのです\footnote{
  実は、正規分布を含む扱いやすい確率分布(\textbf{指数型分布族})においては、標本平均さえ知っていれば、元のデータが持っていたパラメータ推定に関する情報を一切失わない(\textbf{十分統計量}である)ことも知られています。
  この「情報のロスがない」という性質も、平均値が自然な概念として扱われる理由の一つです。
}。
大数の法則および中心極限定理は、測度論的確率論をベースとして証明されるれっきとした定理です。

このように意味では、推定という作業は、無数の可能性(確率分布の空間)の中から、手元のデータを最もよく説明できる「一点(特定のパラメータを持つ確率分布)」を探し出す作業と言い換えることができます。
これは幾何学において、曲面上の点を探す行為に似ています。



\section{ノイズとシグナル}

一般的に中心極限定理より、サンプルが多ければ多いほど、分散が小さくなり、標本平均は母平均に近いと言えます。
しかし「多く」といっても、何件くらいの標本を調べれば満足いく精度の正規分布が分かるかは不確実性があります。
この不確実性を、確率論の力を使って「99\%正しい」とか、「対立する仮定(帰無仮説)は棄却できる」などと断言できるようになります。
このように、標本平均の分散を考えることで、「ノイズ」と「シグナル」を分離することができます。
例えば、
\begin{description}
  \item[実験の分野] 真の法則を発見するために、ある実験結果の膨大なデータ。\\
    $\to$ 結果を人間が目盛りを読んでノートに記述するノイズ、気温や湿度の違い、等。
  \item[天体観測] 観測データのズレがどれだけ大きいか。\\
    $\to$ 同上
  \item[身体測定] 国民全員の身長の測定データ。\\
    $\to$ 測定機器の設備の違い、地域ごとの食文化の違いなど。
  \item[選挙] 出口調査\\
    $\to$ 記憶違い。嘘をつかれる。
\end{description}

ノイズとシグナルを分ける有名な方法としては、\textbf{カイ二乗分布}を使用することが挙げられます。
例えばある実験を30回繰り返すと、その標本の分散が、真の分散と約2倍の誤差の範囲に収まることを99\%保証されます。
逆に言えば、分散が2倍以上離れてしまう可能性は1\%ぐらいに収まります。
実験レポートではここまで計算・分析・考察して報告しなければなりません。



\section{意思決定とリスク管理}

この部分が、統計学の最も役にたつ部分です。
\begin{description}
  \item[実験の分野] 真の法則を発見するために、ある実験結果の膨大なデータ。\\
    $\to$ 結果を人間が目盛りを読んでノートに記述するノイズ、気温や湿度の違い、等。\\
    $\to$ 統計分析によって、ノイズが混ざっている可能性は1\%未満であると言える。
  \item[天体観測] 観測データのズレがどれだけ大きいか。\\
    $\to$ 同上
  \item[身体測定] 国民全員の身長の測定データ。\\
    $\to$ 測定機器の設備の違い、地域ごとの食文化の違いなど。\\
    $\to$ 建築において、ドアの高さの規格が作れる。優位に身長が低い地域があれば、何か健康被害があるのではないかと調査する必要性が生まれる。等。\\
  \item[選挙] 出口調査\\
    $\to$ 記憶違い。嘘をつかれる。\\
    $\to$ 上記のノイズを排除しても20時ちょうどに当選確実の報を出せる。
\end{description}
より具体的には、実験ノイズを排除することで、より普遍的な自然法則に近づくことができるでしょう。
あるいは日本人の身長の中央値を知ることができれば、その中央値よりも少し高いドアを作れば大体良いでしょう。
このことによって、製造メーカーはドアの規格として高さ2mぐらいのドアを作れば、コストに見合う大量生産を行い、大きな利益を得るでしょう。

医療の分野では、特に人の生死がかかわっているので、手術の成功率を統計的手法によって厳密に計算しなければならないです(\textbf{医療統計})。
そして「それでも手術を受けますか?」という提案を医者側からしなければなりません。
このことからも実に重い責務を統計学は担っていることがわかるでしょう。

このように、「真の母集団と標本平均のズレ」がどれだけの確率で発生するかを理論的に検証し、もって産業や医療、あるいは他学問に役立てることができるのです\footnote{
  昔の番組のトリビアの泉では、市場調査で2000人のサンプルを調べれば十分信頼できる結果が得られるというくだりが何度もありました。
}。
統計学では、このような検定方法の理論的裏付けを与えるのです。

特に近年では、ビッグデータ・深層学習・LLMの時代に突入してきており、これまで理論的には無茶とされていたことが覆されてきています。
つまり旧来、深層学習などの分野における素朴な勾配法では、過学習などの問題があり使い物にならないといわれていましたが、この定説を打ち破り、「やってみればうまくいってしまった」という状況なのです\cite{Am25}。
であれば、「なぜうまくいってしまったのか?」という理論的根拠を与えるために、情報幾何学や統計学で発展が急速に進んでいます。
また統計学が情報幾何学を通じて幾何学と結んでしまったため、機械学習と幾何学、量子論、はてはゲージ理論・素粒子論、代数幾何学と結びつき、さらに深い理解が求められるようになってきています。



\section{まとめ}

統計学には以下の大目標があります。
\begin{itemize}
  \item (帰納的)推論
  \item 「ノイズ」と「シグナル」の分離
  \item 意思決定とリスク管理(結局どう統計データを扱うのか?
\end{itemize}

これらを研究することで、産業・医療、はては物理や純粋数学において大いに役立つ学問であることが想像できたでしょうか?
僕たちは、これらを主軸において、統計学の勉強を進めることにします。
