\chapter{統計学}

統計学とは何を目的としているのか?という点について、僕の手元にある教科書ではあまりサクッと触れられているものはありませんでした。
そこで僕は、統計学の基礎を、情報幾何学に向けて理解していきたいと思います。

前提知識としては、ふんわりと測度論的確率論がわかっていて、確率変数の平均値、期待値を計算できることを前提とします。

\section{統計学の目的}

統計学の目標を一言でいうと、「乱雑に集められたデータから、真の特徴を発見し、産業や医療、実験科学の役に立て、もって未来の予測を行うこと」のようです。
例えば実験の分野でいうと、ある実験結果によって得られた結果から、真の特徴を発見し、普遍的な法則とどれだけズレているかを検証することができます。
天体観測でいえば、観測データのズレがどれだけ大きいかを検証することができます。
身体測定でいえば、身体の特徴を発見し、健康管理や病気の予防に役立てたり、建築においてドアの高さをどれぐらいにすれば最大効率の建物を建てられるかというところに役立ちます。
この「真の母集団と標本平均のズレ」がどれだけの確率で発生するかを理論的に検証し、もって産業や医療に役立てようというのが統計学の主な立ち位置です\footnote{
  昔の番組のトリビアの泉では、市場調査で2000人のサンプルを調べれば十分信頼できる結果が得られるというくだりが何度もありました。
}。

まとめると、以下のようになります。
\begin{itemize}
  \item (帰納的)推論
  \item 「ノイズ」と「シグナル」の分離
  \item 意思決定とリスク管理(結局どう統計データを扱うのか?
\end{itemize}

「ノイズ」と「シグナル」は、上記の代表値を用いて適切に分離を行います。
例えば国民全員の身長を調べるのは大変なので、ランダムに100人を選んで身長を測定し、その平均値を国民の身長の代表値として用いることができます。
これを標本平均と言いますが、この標本平均は母平均、すなわち本当の国民全員の身長の平均とは一致しないでしょう。
しかしこれが母平均とどれだけズレているかは、標本平均の分散を考えることでわかります。
分散が小さければ小さいほど、標本平均は母平均に近いと言えます。
このように、標本平均の分散を考えることで、「ノイズ」と「シグナル」を分離することができます。

また、意思決定とリスク管理については、以下のようになります。
TODO: ここを埋める



\section{推定}

基本的には、母集団からいくつかサンプル(標本)の平均値・分散などを取得し、もって母集団の傾向をつかむということを行います。
根底には、十分サンプルを多く取れば、標本平均は真の平均に近づいていくという\textbf{大数の法則}、および、標本分散が真の分散に近づいていくことから、母集団が正規分布をなしているであろうという\textbf{中心極限定理}があります。
したがって、母集団のそのような代表値は、正規分布に従っているだろうという仮定を行う妥当性が生まれます。
対数の法則および中心極限定理は、測度論的確率論をベースとして証明されるれっきとした定理なのです。

しかし多くのといっても、何件くらいの標本を調べれば満足いく精度の母集団についての正規分布が分かるかは不確実性があります。
この不確実性を、確率論の力を使って「99\%正しい」とか、「対立する仮定(帰無仮説)は棄却できる」などと断言できるようになります。



\section{ノイズとシグナル}

天体観測や実験等において、その観測結果が真の自然法則が示す値に近いかどうかを統計学によって計算することができます。
例えば中和滴定の実験で、10ml加えると中性に傾いたというのは、人が目盛りを見て判断していますが、当然そこにノイズが含まれてしまいます。
カイ二乗分布を使用すると、例えばその実験を30回繰り返すと、その標本の分散が、真の分散と約2倍の誤差の範囲に収まることを99\%保証します。
逆に言えば、分散が2倍以上離れてしまう可能性は1\%ぐらいに収まります。
このことを実験レポートとして報告しなければなりません。

統計学では、このような検定方法の理論的裏付けを与えるのです。



\section{意思決定とリスク管理}

この部分が、統計学の最も役にたつ部分です。
例えば日本人の身長の中央値を知ることができれば、その中央値よりも少し高いドアを作れば大体良いでしょう。
このことによって、製造メーカーはドアの規格として2.5mぐらいのドアを作れば、コストに見合う大量生産を行い、大きな利益を得るでしょう。

例えば医療の分野では、特に人の生死がかかわっているので、手術の成功率を統計的手法によって厳密に計算しなければならないです(\textbf{医療統計})。
そして「それでも手術を受けますか?」という提案を医者側からしなければなりません。
このことからも実に重い責務を統計学はになっていることがわかるでしょう。

近年では、ビッグデータの時代に突入してきており、これまでの統計学のように、手計算では対応できないほどのデータ量が集積されています。
これらのデータを解析するために、古来より様々な統計学の手法が開発されてきました。
旧来といえば、深層学習などの分野における素朴な勾配法では、過学習などの問題があり使い物にならないという定説を近年打ち破り、「やってみればうまくいってしまった」という状況となってます。
であれば、「なぜうまくいってしまったのか?」という理論的根拠を与えるために、情報幾何学や統計学で発展が急速に進んでいます。
また統計学が情報幾何学を通じて幾何学と結んでしまったため、機械学習と幾何学、量子論、はてはゲージ理論・素粒子論と結びつき、さらに深い理解が求められるようになってきています。



\section{まとめ}

統計学には以下の大目標があります。
\begin{itemize}
  \item (帰納的)推論
  \item 「ノイズ」と「シグナル」の分離
  \item 意思決定とリスク管理(結局どう統計データを扱うのか?
\end{itemize}

特に最後の「意思決定とリスク管理」を行うための手段として、ビッグデータに対する「推論」と「検定(ノイズとシグナルの分離)」を行うのです。
そして推論を検定を行うベースに大数の法則と中心極限定理があり、公理論的確率論によりこれらは厳密に証明された定理なのです。
